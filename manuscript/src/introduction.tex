Atacama Large Millimeter/Sub-Millimeter Array (ALMA)~\cite{2010AAS...21547901H}
is a joint project between astronomical organizations of Europe (ESO), North
America (NRAO), and Japan (NAOJ). ALMA is a large radio-astronomical project,
it will consist of at least 50 twelve meter antennas operating in the
millimeter and sub-millimeter wavelength range with baselines up to 10\,[km].
It will be located at an altitude above 5000\,[m] on the Chajnantor plateau in
the middle of the Chilean Atacama desert. The science commissioning of ALMA
will start in 2012 when the array will be fully operational for astronomical
observations.

ALMA Common Software
(ACS)~\cite{Chiozzi02,Chiozzi04,Raffi01,chiozzi09:_acs_status} is an Object
Oriented CORBA middleware framework (for science facilities) that
handles communication between distributed objects.  ACS was built to support
the complex control requirements of ALMA radio telescopes, but can be used to
support control and data flow for any system with similar performance
requirements~\cite{gchiozzi02}.
Particularly, it features a Sampling System~\cite{acssamp}, which is used to
continuously collect the values of an indicated set of properties around the
system. This data can then be displayed by a graphical client (the Sampling
System GUI), in form of a
dynamical plot. This client is written in Java, using the JChart2D library, and
is currently being used at the ALMA observatory. Nevertheless, a question
remains open: Which is the best library/language combination for such a
task?
One point to note is that ACS is open source
(it is distributed under LGPL),
so any base software used in it
(like graphics libraries)
must be compatible with this.

%\subsection{Trending in a Distributed System}
The work presented in this paper
aims at evaluating different alternatives for high performance
graphical data trending in
distributed systems.%, ALMA is an excellent example of a distributed environment.
It is important to distinguish trending
(contructing graphs of data flowing in real-time)
from plotting
(where graphs,
 possibly very complex,
 are constructed from statically available data).
Trending is inherently real-time,
and is mostly concerned with showing trends for data that evolves in time.

First the most common  data trending solutions were identified, and a list of
alternatives is described. Performance tests are applied to the tool
developed for the ALMA project in order to asses the
performance of different graphical libraries.
The problem of selecting a programming language and trending tool
are described,
then we discuss the state of the art
in trending tools and graphical trending libraries.
A methodology to test performance is then proposed,
and the resulting benchmarks are discussed.
Some real-case scenario tests were performed.

%%% Local Variables:
%%% mode: latex
%%% TeX-master: "../article"
%%% End:
