% General Conclusions

Better ways of measuring performance,
like the number of data streams that can be displayed at a given FPS
or the impact of smoothly evolving data or scattered points,
should be considered.

Looking backwards, the evolution of the different libraries give the
programmers a lot of ``path to follow'' when starting a new project. The performance
of the actual data trending libraries in different languages is very important
because all of them try to offer ``simple programming'' in this application area.
Comparing the features of the libraries is not so important, because most of them
are rather robust, and those with more features just allow creating more elaborate plots.
Without going further, there are other characteristics to consider, like
``perdurability,'' ``modularity,'' and ``scalability'' of software.
For a long-lived project like ALMA
(and its supporting software ACS)
these characteristics are crucial.
How to estimate the stability and longer range development and maintenance
of open source software on which something like ACS is based
is still very much an open question.

In section~\ref{sec:gbenchmark} we saw benchmarks of a couple of graphic libraries
in the three languages that ACS uses, C++, Java, and Python.
The benchmark shows the behavior of these libraries in three levels
of stressed environments, giving us data to discriminate among them.
Anyway, this is not a final decision, because one benchmark is not enough to test the real
performance of a graphical library, but it is a good start to help discriminating them with the previous
basic functionality (plotting random data).
Also,
other important characteristics haven't been considered in detail.
For a long-lived protect like ALMA
(and its supporting ACS package)
using stable, well-maintained base software is crucial.

Finally, in section~\ref{sec:benchmark} we can see the performance of
the existing scientific data visualization tool, developed by the Computer Systems Research Group (CSRG),
and giving us the chance to analyze a real application in a real distributed system such as the
ALMA project.

%%% Local Variables:
%%% mode: latex
%%% TeX-master: "../article"
%%% End:
