% The performance of the language
When a development team is given the task of creating a software project,
one of the first questions is the selection of the programming language.
This is a very important decision, as performance is greatly affected by the
paradigm and implementation of the available compilers for that programming
language. The usual manner to solve this is to prototype, use previous
experience and test the different wanted characteristics, such as
modularity, performance, and scalability.

There are comparisons between different programming
languages, such as~\cite{empirical}, in which the author gives a walk
through the origin of each language, and puts emphasis in the validation
of a programming language comparison, because that depends on different
characteristics, such as programmer capabilities, the different task
and work conditions, the handling of misunderstood requirements,
the paradigms employed (object oriented, imperative, generic programming).
As is to be expected, such comparisons tend to concentrate only on one area.

% performance of the graphical library

Aside the selection of programming language, the team has to
consider the selection of a graphical library. This is a nested decision, as very
few libraries are available for several languages.
Then, the question arises: ``Of all the graphical libraries available, which is
the best one for my project?''
Strictly speaking, one should analyse a couple of libraries and
then make a decision. In practice this is almost never done, as schedules are usually tight so
there is no time available to evaluate every choice.

In our case, the main problem is to find the best choice in the data trending
area, where performance of the graphical representation are critical.
There are several ways to measure the performance and quality of graphical
libraries, some examples are:
\begin{itemize}
	\item Number of chart types handled.
	\item Trace options.
	\item Plot functionality.
	\item Frames Per Second (FPS).
        \item Data volume vs. performance
\end{itemize}
In the present work the most important metrics are FPS and the data
volume vs. performance. Large amounts of data need to be quickly processed
and displayed, together with additional information.

% operations in distributed systems

The development team has still two decisions open, both of them pending from a
proper comparison. But there is yet another problem.
Many systems in real world applications are not \emph{single-node} systems,
so that performance tested on a single computer may give a different result
than when deployed on a distributed system. This is the case of
ALMA and ALMA Common Software (ACS) distributed applications. Distributed systems are complex, and
communications is an important part of the data pre-processing.

% The idea is have severals autonomous computers, communicating with each to
% another,
% trough a computer network in order to achieve a common goal.
% Generally this systems are like \emph{house of cards},
% all the cards are important, without one, the system will not work.

Given this, a third question remains open: ``How does a distributed system
affect my trending application's performance?''

% data trending in distributed systems

% Finally, the data trending per se needs a good choice of the previous
% statement,
% to achieve a high performance application.

%%% Local Variables:
%%% mode: latex
%%% TeX-master: "../article"
%%% End:
