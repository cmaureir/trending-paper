Currently, a huge range of plotting solutions exists in the market. As time
passes, more and more libraries are developed, or existing ones are improved,
extended or updated. This makes it difficult both choosing one of such libraries,
and to compare them in a distributed environment.
In this section we aim to present some of the most used plotting and data
trending solutions. This list is constrained to the set of languages that are
used in ALMA Common Software (i.e., Java, C++ and
Python).
They are of help in understanding which
variables are important to measure when considering the election of one
solution.

% graphical libraries state of art (Cristian)
\subsection{Graphical Libraries}

The most significant part of the work when plotting data collected over a
distributed system is the plotting itself. In the following section we present
some of the most used plotting libraries, giving an overview of their
implementations.

\subsubsection{Java}

% JFreeChart
\emph{JFreeChart}~\cite{JFreeChart} is a very popular chart library, since it allows to create complex charts easily.
One of the most important characteristics, from the programmer's point of view, is its well-documented API.
Currently, \emph{JFreeChart} supports different chart types, such as X-Y charts (line, spline, and scatter),
pie charts, Gantt charts, bar charts (either horizontal, vertical and stacked and independent), and finally
a single value graph (like for the case of a thermometer, compass, or speedometer).
The flexible design of this library allows to extend the application in both the server and client sides.
It also supports many output formats, including Swing components, image files, and
vector graphics.
Finally, \emph{JFreeChart} is open-source software, distributed under the terms of the GNU Lesser General
Public Licence (LGPL), which allows to use it both in open-source and proprietary applications.

% JChart2D
\emph{JChart2D}~\cite{JChart2D} is a charting library designed for displaying
multiple \emph{traces}, which in turn consist of \emph{trace-points},
Its main advantage is that it provides the programmer a minimalistic way to work with charts.
It is important to note that \emph{JChart2D} is centered around a Swing widget (Chart2D),
so knowledge of Java AWT and Swing technologies is very useful.
Some of its features are the renderization of the traces via lines, discs, dots, or filled polygons, multiple axes on top,
bottom, left and right side, zoomable charts, multiples traces with different behavior in a simple chart,
a toolbox of UI controls for charts via pop-up menus, automatic choice of units, optional display of grids, labels,
and more.
\emph{JChart2D} is intended for engineering tasks, since its speciality is the dynamic precise display of the data in
a minimalistic way, without much configuration.
\emph{JChart2D} is also published under LGPL.
% OSI Aproved? What is that?

%JCCKit
The \emph{JCCKit}~\cite{JCCKit} is a very small chart library, which features a very flexible framework for creating
scientific charts and plots with the necessary elements.
\emph{JCCKit} is suitable for scientific applets, because it is written for the JDK 1.1.8 platform.
% Wow, it sounds pretty old then...
The purpose of this library is to provide a flexible kit for programming applets and application for the visualization
of some data.
Some important features of \emph{JCCKit} are that it is highly configurable,
its extensibility and
customizability, the automatic update of changed data, dynamic charts and plots, automatic
rescaling, good support for logarithmic axes, line styles, symbols, fonts,
error bars, and the compatibility with AWT graphic, SVG, and more.

%QN Plot
\emph{QN Plot}~\cite{QNPlot} is a chart implementation that provides a way to program graphs (of one or more functions)
as Swing components. Its design makes it possible to render large amounts of real-time data,
which is an advantage when it comes to its usage in distributed systems.
Some features of \emph{QN Plot} are the coordination of different kinds of big decimals having arbitrary precision,
its high performance with large amounts of data, all the classes in its implementation are thread-safe,
the schemes of the axes have been specially written to choose step sizes for the index automatically.
Finally, \emph{QN Plot} is free software, released under the 2-clause BSD license.

\subsubsection{Python}
% PyQwt
The \emph{PyQwt}~\cite{PyQwt} module is a set of Python bindings for the Qwt C++ class library which extends some
features of the Qt framework to be able to build widgets for scientific and engineering applications.
This library provides widgets to plot 2-dimensional data and work with the  control of bounded or unbounded floating point values, and very large integer values, with various widgets displaying them.
The main idea of \emph{PyQwt} is to merge some Python modules, so as to have a complete framework to work with
data. It mixes the GUI library PyQt, the plotting library Qwt, and NumPy and SciPy to cover the mathematical data
manipulation because those libraries have a lot of computational methods that help manipulating
the data easily, because the combination of NumPy and SciPy offers an environment very similar to Matlab.

% Matplotlib

\emph{Matplotlib}~\cite{Matplotlib} is a very powerful plotting library developed for the Python programming language, which produces high quality figures in different formats.
One of the main goals of this library is to make it easy to plot and manipulate data, because in a few lines one can generate histograms, bar charts, scatter-plots,
simple plots, etc.
Besides the Python language, \emph{Matplotlib} uses NumPy, a numerical mathematics extension for Python, to do all the heavy mathematical processing.
\emph{Matplotlib} is very similar to MatLab, because it offers the PyLab interface to simplify learning, principally to the MatLab user. This makes it
a good choice for numerical mathematics and signal processing tasks.
Finally, \emph{Matplotlib} is distributed under a BSD-style license.


% Biggles

Looking at other Python libraries, we found \emph{Biggles}~\cite{Biggles},
which provides a lot of useful tools to create and manipulate scientific plots, having too as main idea
the complete customization of the plots, so for \emph{Biggles} the plots are a set of very simple objects.
The idea of the \emph{Biggles} objects is a good classification, taking two categories of objects, the containers and the components; so when we have a container we
may use a lot of components for that container.
Finally, we have the concept of Container for all the plots, tables, etc and the idea of the Components that need a Container to work, and can't be visualized on their own.
\emph{Biggles} is a new graphical library, so it has far to go. This library is distributed under the terms of the GNU General Public License.

% PyQtGraph

\emph{PyQtGraph}~\cite{PyQtGraph} is a very good library, because it uses two very popular Python modules and combines them in a nice way, we are talking of PyQt and NumPy,
so the idea of \emph{PyQtGraph} is to combine all the features of NumPy with the widgets provided by the PyQt wrapper.
The objectives of this library are to be a good library to work with mathematics, scientific, and engineering applications.
\emph{PyQtGraph} is a very fast library, because it is written purely in Python and uses all the numerical capacities of NumPy and the fast display
of Qt applications.
Beside all the widgets that \emph{PyQtGraph} provides, there are two important features, the highly feature-rich plotting systems and an image display system
with region-of-interest widgets.


\subsubsection{C++}
% Qwt
Previously, we mentioned the PyQwt library, and we said that it is a wrapper for \emph{Qwt}~\cite{Qwt}, while \emph{Qwt} is an extension to the Qt library that
contains useful GUI Components and utilities.
Its main feature is the 2D plot widget, but \emph{Qwt} provides several widgets/components that facilitate programming.
Some of these components are scales, sliders, dials, compasses, thermometers, wheels and knobs to control or display values, arrays, or ranges of type double.
\emph{Qwt} is distributed under the terms of the Qwt License, a variation of the GNU LESSER GENERAL PUBLIC LICENSE (LGPL) with some exceptions.

% Koolplot

\emph{Koolplot}~\cite{Koolplot} is a very simple-to-use library that allows to create and manipulate 2D graphs.
It is very small and basic, so it is not recommended for use in complex graph situations.
A feature of \emph{Koolplot} is the compatibility with the MinGW compiler, so it can be used on Linux and Microsoft Windows platforms.
Finally, \emph{Koolplot} is in the public domain.

% wxMathPlot

\emph{wxMathPlot}~\cite{wxMathPlot} is a properly built library to add 2D plot scientific functionality into wxWidget, a cross-platform C++ library to create applications
for Microsoft Windows, OS X and Linux.
As it can be inserted into wxWidgets, it allows to embed inside every wxWidget application a window for plotting different types of data.
Some of the most important features of \emph{wxMathPlot} are the completely mouse-driven view control (pan, zoom, scroll, etc), the different output formats of
the screenshots (BMP, PNG and JPEG), the flexible axis positioning, the several layers to plot data from vectors, movable objects, bitmaps, etc.
Finally, \emph{wxMathPlot} is distributed under the therms of the wxWindows Licence, that is essentially LGPL with some exceptions.

% CARNAC
\emph{Carnac}~\cite{CARNAC} Chart Library is an extension to the Qt library that adds powerful visualization.
The main idea is to allow programming complex charts with minimum effort.
Some features of \emph{Carnac} are the large number of different chart types supported,
and its flexibility in setting up axes and labels.

% GNUplot++

\emph{GNUplot++}~\cite{GNUplot++} is a wrapper of GNUplot through C++.
GNUplot is a very old and properly build command-line program that can generate different types of plots, frequently used for publication quality graphics,
and is multi-platform (Linux, Microsoft Windows, Mac OS X, etc)
\emph{GNUplot++} mixes the powerful GNUplot tools with many features of standard C++, for example templates class.
It uses many features of standard C++, like integration to the standard template library (STL) and its iterators.
\emph{GNUplot++} is distributed under GPL.

%%% Local Variables:
%%% mode: latex
%%% TeX-master: "../article"
%%% End:
